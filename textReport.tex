
%%%%%%%%%%%%%%%%%%%%%%%%%%%%%%%%%%%%%%%%%%%%%%%%%%%%%%%%%%%%%%%%%%%%%%%%%%%%%%%
%                                                                             %
%  1. Introduction                                                            %
%  \label{sec:introduction}                                                   %
%                                                                             %
%%%%%%%%%%%%%%%%%%%%%%%%%%%%%%%%%%%%%%%%%%%%%%%%%%%%%%%%%%%%%%%%%%%%%%%%%%%%%%%

\fontsize{12}{12}\selectfont
\section{Introduction} \label{sec:introduction}
%
	During the last few decades, the formation and the evolution of early-type galaxies has been a much debated topic 
	in astronomy. 
	%
	The main stellar component of these objects is old and spans a range in mass from $10^7$ to 
	$10^{12}~M_{\rm{\odot}}$. 
	%
	Hosting little or no star formation, elliptical and lenticular galaxies are perfect cases to study in order to 
	understand the scenario in which they formed. 
	%
	Until a few years ago, the two main formation scenarios considered were \textit{monolithic collapse} and 
	\textit{hierarchical merging}. 
	%
	The first theory assumes that early-type galaxies formed in a single and violent burst of star formation at high 
	redshift, followed by a quiescent history in which they didn't experience any other star formation episodes 
	\citep{Larson74, Carlberg84, Arimoto87}. 
	% 
	On the other side, the \textit{hierarchical} model considers the bigger galaxies to have formed via mergers of 
	smaller objects continuously through out the galaxies' lifetime \citep{Toomre72}. 
	%
	These mergers can be either gas-rich (wet mergers) or gas-poor (dry mergers). 
	%
	While in the latter case the original stellar populations of the merging galaxies will be mixed together in a dissipationless 
	process (i.e. without new stellar formation, which would dissipate energy), in the former the accreted gas can 
	trigger new star formation, producing an additional third stellar population.
	%

	Unfortunately in the few last years both the \textit{monolithic collapse} and the \textit{hierarchical merging} scenarios have 
	been proved to be unsuitable in their ``pure'' form by the observations.
	%
	In the first case, for example, it is not possible to explain the significant size evolution observed for the massive galaxies over 
	cosmic time without postulating a continuous accretion of minor satellites \citep{Daddi05, Trujillo06}. 
	%
%	In the first case, for example, it has been observed that galaxies at high redshifts are less massive than at the present day, 
%	in contrast with the hypothesis of massive galaxies forming earlier because of their stronger gravitational potential. 
	%
%	On the other side, gas rich mergings can not explain the very old stellar population present in the early-type galaxies, which also 
%	does not show the presence of cold gas or star formation in the nearby universe. 
	%
	On the other side, semi-analytical simulations shown that it is not possible to reproduce just through mergers the observed 
	abundances of massive galaxies 	in the local universe \citep{Conselice07}.


	%HYBRID MODELS
	For this reason, other models have been proposed in recent years. 
	%
	\citet{Oser10} present galaxy formation and evolution as a two phase process, in which 
	both the original scenarios are partially true. 
	%
	In this case the bulk of early-type galaxy stars are formed early through dissipative collapse of gas, as predicted by the 
	\textit{monolithic} picture, but subsequent growth in size during the galaxy lifetime is due to the accretion of 
	smaller satellite galaxies. 
	%
	Unfortunately, the increasing in size and mass observed in massive galaxies after $z\approx2$ requests a number 
	of satellites around massive galaxies that is not consistent with what found by observations \citep{Quilis12}.
	%
	At high redshifts the formation of the initial system of stars could also be formed by the instability of a 
	clumpy disk of gas \citep{Dekel09}.
	%
	The gas from the external halo flows in a rotating disk via cold streams, forming clumps where the star formation is 
	enhanced and depositing its angular momentum. 
	%
	These clumps then migrate toward the centre because of the dynamical friction with the gas of the disk, where they 
	start to build a spheroid. 
	%
	If the clumpy component of the streams is small enough, the migration is dominant on the creation of new clumps in 
	the disk and this disappears. 
	%
	If, on the contrary, the clumpy component is higher, the disk will keep growing.
	%
	After $z\approx1$ the galaxies become stable because the cosmological accretion rate becomes slower and the flows of gas 
	are not able to enter the shock-heated halo anymore, cutting the gas support to the disk. 
	%
	This may explain the compact, early-type galaxies seen at $1.5 < z< 2.5$ which today appear to have disks surrounding 
	them \citep{Graham11, Dullo13}.
	%
	The early disks might be the present day thick disks or lenticular galaxies, or form ellipticals via mergers. 
	%
	In this case the central spheroid will continue to grow by slow secular evolution of the disk.
	%
	If the disk wasn't present at the moment of the stabilization, the spheroid will then grow only by minor and major 
	mergers, forming the present-day elliptical galaxies. 
	%
	Both the \citet{Oser10} and the \citet{Dekel09} models join together elements of the \textit{monolithic} scenario 
	(i.e. bulk of stars formed at high redshifts) with assumptions of the \textit{hierarchical} scenario (i.e. the 
	importance of mergers in the evolution of the spheroids). 	%DUNCAN DICE CHE NON E' VERO NEL WILD DISK, INVECE SI
	%
	For this reason they are often considered as \textit{hybrid} scenarios, in which the importance of the initial formation 
	against the later accretion is still an object of debate.	
	
	%Metallicity
	A way to better understand the mode in which the present-day galaxies formed and evolved is to compare the 
	predictions of the models with data from observations. 
	%
	In particular, the one verifiable prediction is the chemical composition of the present-day galaxies. 
	%
	In fact, the ``pure'' \textit{monolithic} and \textit{hierarchical} models provide two different behaviours for the 
	distribution of metals in the galaxy.
	%	
%DEFINITION METALLICITY
%	The lifetime of a star depends mostly on its mass. 
%	%
%	Very massive stars turn into supernovae very soon, spreading their chemically-enriched content in the surrounding 
%	space. 
%	%
%	This matter is turned into new stars, which will therefore have an higher content of high atomic-number elements, 
%	/..
	In the first case, the gas tends to fall toward the centre of the gravitational potential well, where it is continuously 
	enriched by evolving stars. 
	%
	As a consequence of this infall, stars formed in the centre are more metal-rich than stars in the outer galaxy 
	regions, and a negative metallicity gradient towards larger radii is set \citep{Chiosi02}. 
	%
	The efficiency of this process is increased by the presence of stronger gravity potential, so both the steepness 
	of the metallicity gradient and the central metallicity \citep{Spolaor10b} are expected to increase 
	with the mass of 	the galaxy (i.e. the deeper the gravitational well, the more metal-rich the central stars are). 
	%
	
	%
	The expectations from the \textit{hierarchical merging} model are quite different. 
	%
	If the formation of a present-day elliptical galaxy is mostly driven by multiple gas-poor mergers, for each one of them 
	the mix of differently formed stellar populations would flatten the resulting metallicity radial profile. 
	%
	Conversely, if the mergers are rich in gas, this would move toward the centre of the gravitational well, enhancing 
	new star formation. 
	%
	In this latter case the steepness of the metallicity gradient is increased by the presence of a chemically enriched new 
	stellar population \citep{Pipino08}, resembling the expectations of the 
	\textit{monolithic collapse} scenario in the central part of the galaxies.  
	%, although is not possible to completely 
	%replicate in this way the steepness from the collapse scenario \citep{Kobayashi04}. 
	%
	In the outskirts, instead, the pre-existing stellar populations of both the galaxies are mixed and the 
	expected radial profile is flat.
	%
	
	%	
	Because of the same inner metallicity profile expected in both the wet mergers and primordial accretion scenarios, it 
	is thus 	evident that a study of the stellar populations in the outer parts of the galaxies is necessary to discern between 
	the two. 
	
	%In the Section \ref{sec:metAn}, the analysis of the metallicity will be addressed in more detail, while in Section \ref{sec:IMF}  
	%the Initial Mass Function problem will be presented and discussed.	
			
\subsection{Metallicity analysis}	\label{sec:metAn}
	As shown in the first part of this Section, %\ref{sec:introduction}, 
	the study of the stellar metallicity $[Z/H]$ in the centre and in the outskirts 
	of a galaxy can lead to an understanding of the formation and evolution mechanisms that a galaxy experienced. 
	%
	This is because the metallicity of the integrated stellar component is linked with the formation and the later 
	evolution of the galaxy.
	%
        
        A problem in the study of the metallicity from just a photometric analysis is the ``age-metallicity'' degeneracy. 
        %
        In fact, both the age and the metallicity of a stellar population affect the colour of the galaxy in the sense that 
        old stars as well as metal-rich stars tend to have redder colours. 
        %
        The only way to infer the correct chemical composition is, thus, from a spectroscopic analysis of the stars.
	%
        In particular, from an analysis of the integrated stellar population spectrum in the optical wavelenghts, the Lick/IDS 
        system \citep{Worthey94} was defined to overcome the degenaracy problem. 
	%
	This system of spectroscopic indices considers different chemical absorption lines, more or less dependent on either the age of the 
	stellar population and its metallicity, to separate the two properties. 
	%
%	The lines that have been most used in the past to extract the metallicity are all the metal lines, in particular 
%	the many lines at various wavelength of the iron. 
	%
	The Lick indices were originally obtained from lines in the optical range of the spectrum, but they have been 
	recently integrated with indices in the UV range \citet{Franchini10} in order to be usable with the 
	data from the Sloan Digital Sky Survey (SDSS). 
	%
        While the Lick system allows us to retrieve a reliable estimation of the metallicity, it is not useful to obtain 
        information about the initial mass function (IMF) of the probed stars. 
        %
        In particular, unsuitable for studies in the near infrared region, the Lick system does not probe this part 
        of the spectrum where the light of giant stars dominates. 	
	%
	
	Luckily a very useful spectral feature in this spectral range exists. 
	%
	As shown in \citet{Cenarro02}, the near-infrared calcium triplet (CaT) feature strongly correlates with the metallicity 
	of the stars and can be assumed to be unaffected by the age of the stellar population if this is 
	older than $3~\rm{Gyr}$ \citep{Vazdekis03}. 
	%
	An index based on this spectral feature was first used by \citet{Armandroff88} to determine the metallicity of Milky Way 
	globular clusters and has later been successful in determining metallicities for individual stars \citep{Rutledge97, Battaglia08}. 
	%
	Recently, \citet{Usher12} compared the metallicities obtained from the CaT index with those from Lick indices 
	for the globular cluster systems of 5 galaxies, finding a good correspondence with an rms scatter of $0.27$ dex. 
	
	%CaT Puzzle
  	The study of the CaT strength has also presented the astronomers an interesting conundrum, which it has been 
  	often referred to as the ``CaT puzzle'' \citep{Michielsen07}. 
	%
	In fact, despite other metallicity tracers such as the $\rm{Mg}_2$ strongly correlating with the mass of the 
	host galaxy, it has been found that CaT anticorrelates with the mass of elliptical galaxies 
	\citep{Saglia02, Cenarro03, FalconBarroso03}. 
	%
	Interestingly, this anticorrelation has been proven to exist down to the dwarf elliptical mass regime 
	\citep{Michielsen03}. 
	%
	During these years several different solutions have been proposed to explain this anticorrelation, such as 
	the possibility that the assumption of a single stellar population for these galaxies 
	was wrong \citep{Pasquali05} or the possible link between the calcium abundance with the mass of the 
	galaxy. 
	%
	Unfortunately, most of them are not able to satisfactorily explain the CaT variation with the galaxy 
	mass. % without requesting a strong fine-tuning. 
	%
	An exception in this sense is the hypotesis that the CaT could be linked with the IMF of the galaxy in which 
	it is measured. 
	
	%
	%ISSUE WITH THE IMF, GCs have a Kroupa (bottom light).
	%
	
	%HOW TO STUDY METALLICITY
%	%
%	In the past, the metallicity studies were carried out mostly based on longslit spectral observations 
%	\citep{Davies93, Sanchez-Blazquez07, Spolaor09}. 
%	%
%	Most of them needed long exposure time in order to retrieve a 	signal-to-noise (S/N) ratio high enough to 
%	extract the faint absorption lines linked with the metallicity.
%	%
%	For this reason, the metallicity profiles available in the literature are most exclusively extracted from just the major 
%	and the minor axis of the galaxies and within $1~R_{\rm{eff}}$. 
%	%
	
	%IFU
%	With the develop of integral field unit (IFU) spectrographs, the approach to the problem drastically changed. 
%	%
%	Now, the metallicity could be extracted from various different regions of the galaxy field and the necessary 
%	S/N ratio could be obtained much easily by integrating the probed regions along the isophotes \citep{Rawle08, Rawle10}. 
	%
%	One of the most efficient instruments in this sense is the ``Spectroscophic Areal Unit for Research on Optical Nebulae'' 
%	(SAURON) \citep{Bacon01, deZeeuw02} which observed initially 72 nearby galaxies and is currently the instrument of the 
%	much more extended $\rm{ATLAS^{3D}}$ project \citep{Cappellari11}. 
	%
%	A limitation of many IFU instruments is their limited field-of-view that can barely probe distances from the centre of 
%	nearby galaxies greater than $1~R_{\rm{eff}}$ \citep{Rawle10, Kuntschner10}. 
	%
\subsection{Initial Mass Function}\label{sec:IMF} 
%	A fundamental quantity to study galaxy formation and evolution is the initial mass function (IMF) which describes 
%	the mass distribution of a stellar population at the time of its formation (i.e. without involving stellar evolution).
	%
	In the framework of galaxy formation and evolution, knowledge of the IMF is necessary to correctly interpret the 
	integrated stellar spectra we observe and to retrieve the stellar populations from them. 
	%
	This is because stars with different masses will evolve differently, modifying the surrounding environment in which 
	new generations of stars will eventually form. 
	%
	While low-mass stars ($M<1~M_{\rm{\odot}}$) fuse H to He on timescale comparable with the Hubble time, 
	higher-mass stars have a much faster evolution and, expelling a large fraction of their mass at the end of their 
	cycle, they enrich the ISM with heavy elements. 
	%
	The only way to directly constrain the IMF is still the study of the stars in the Milky Way disk. 
	%
	Such study was conducted on the solar-neighbourhood stars by \citet{Salpeter55} in which 
	a power-law relation with index $\alpha=2.35$ was found to describe the histogram of 
	stars with masses in the range $0.4-10~M_{\rm{\odot}}$.
	%
	
	Indirect methods to study the IMF involve the mass modelling of the stellar system \citep{Cappellari06, Thomas11} or the 
	spectroscopic analysis of the stellar light \citep{Cohen78, Faber80}. 
	%
	While the first method compares the total mass with the stellar mass-to-light ratio of the system, the second method takes advantage of 
	the link between several line indices and properties of the stars such as age, metallicity and surface gravity. 
	%
		
	Thanks to these, in recent years the existence of a universal IMF has been disproved by many studies \citep{Bastian10, Cappellari12}. 
	%
	In particular, a strong correlation between the relative number of low-mass stars and the total mass of the galaxy 
	has been found in elliptical galaxies \citep{Treu10, vanDokkum11, Dutton12, Conroy12} and in ultra faint dwarf galaxies \citep{Geha13}.
	%
	From the analysis of three IMF-sensitive spectral lines from the integrated stellar population in early-type galaxies, 
	\citet{Ferreras13, LaBarbera13} extracted an empirical relation between the steepness of the IMF at low 
	masses ($\mu_{\rm{IMF}}$) and 
	the central velocity dispersion ($\sigma_{\rm{0}}$) of galaxy (which links with the total stellar mass of the system). 
	%
%	\begin{eqnarray}
%		\mu_{\rm{IMF}} = 4.87 \log \sigma_{\rm{0}} +1.33.
%	\end{eqnarray}	 
	%
	%PAPER La Barbera 2013 submitted
	%
	Unfortunately, finding the correct slope and shape of the IMF is a problem of an extraordinary physical complexity, within which 
	many variables are involved. 
	%	
	An example of this is the additional dependence that has been found between the IMF slope and the metallicity of the 
	environment in which the stars formed \citep{Kroupa02}. 
	%	
			
	The presence of absorption lines sensitive to the presence of different kinds of stars can however set 
	better constraints on the problem. 
	%
	This is the case of the near-infrared absorption lines, which lie in a region dominated by the light of giant stars and can, thus, 
	be linked with the steepness of the IMF. 
	%


%\section{Thesis project}
%%
%	The purpose of my Ph.D. thesis is to study of the stellar kinematics and metallicities in the outer regions of the elliptical galaxies. 
% Combining recently developed data reduction techniques and statistical 2D interpolation methods I will be able to retrieve reliable stellar chemo-dynamical maps up to large galactocentric radii. My work will improve the understanding of which formation scenario can be taken in account to describe how the early-type galaxies form, investigating the answers to these key questions:
%• Are the kinematic signatures in the outskirts of early-type galaxies suggesting a history dominated by mergers or by quiescent evolution?
%• What is the chemical structure of the early-type galaxies at large radii and how can it help in tracking the enrichment history of these objects?
%• What is the mass function of accreted satellite galaxies?


\section{First thesis project: mapping the metallicity in the outskirts of galaxies}
%
	In my first project, metallicity maps were presented for a range of nearby early-type galaxies.
	%
%THE SLUGGS SURVEY
	The 18 observed galaxies are part of the SAGES Legacy Unifying Globulars %The sample
	and Galaxies Survey (SLUGGS).
	%
	This survey benefits from the Keck/DEIMOS spectrograph by obtaining high resolution spectra of globular clusters (GCs)
	and stars across a wide field of view. 
	%
	The wavelength range of the DEIMOS multislit spectrograph encloses the near-infrared calcium triplet absorption 
	feature, which furnishes both the kinematics and the metallicity information. 
	%
	We have been able to do this out to $10~R_{\rm{eff}}$ for the GCs and 
	up to $3~R_{\rm{eff}}$ for the integrated stars of the galaxy. 
	%
	The aim of this survey is to measure halo properties such as mass, angular momentum and metallicity gradients, 
	orbit structure and substructures, properties which provide important clues to the assembly history of galaxies. 
	%

%THE SMEAGOL PROJECT
	While the survey was initially designed to study the globular cluster population in the galaxies, 
	from the same raw dataset it became possible to extract also the stellar light spectra from very  
	large radii. 
	%
%	This project inside the main SLUGGS is named Spectroscopic Mapping of Early-type Galaxies to 
%	their Outer Limits and uses DEIMOS to extract this information with the Stellar Kinematics with 
%	Multiple Slits (SKiMS) technique. 
	%

	
	\subsection{Data reduction}
	
%The method (SKiMS)-> metallicity

%How we are studying metallicity (SKiMS)
        The slits designed in our DEIMOS program were targeted on globular clusters around the galaxies in our sample. 
        %
        Each slit includes the light from three different sources: the targeted object, the atmospheric foreground light 
        and the background galaxy stellar light. 
        %
        Thanks to the IDL SPEC2D reduction pipeline \citep{Cooper12, Newman12}, the first component is separated 
        from the other two. 
        %
        From the remaining spectrum we extracted the galaxy stellar component using the technique described in  
        \citet{Proctor09} and \citet{Foster09}. 
        %
        This method returns the near infrared spectra for the galaxy stars out to several effective radii. 
        %
        Because of the rapid decline of the stellar flux with radius, the S/N ratio decreases accordingly. 
        %
        While for kinematic purposes one can obtain reliable spectra out to $5-6~R_{\rm{eff}}$ \citep{Arnold13}, to 
        obtain the metallicity we need a $S/N > 35$, which limits our measures to $R \leq 3R_{\rm{eff}}$. 
        %
        Even with this constraint, we are able to chemically explore regions of the galaxy never studied 
        before. 
        %
        Finally, from each of these spectra we retrieved the metallicity $[Z/H]$ from the Calcium Triplet (CaT) lines.

	%Defining the calcium triplet (CaT) index as a weighted sum of the near-infrared calcium triplet equivalenth 
	%widths, we converted it in a value of metallicity for each spectrum. 
	

	\subsection{Metallicity and IMF correction}


%Metallicity conversion issues.
	The three absorption lines of the Calcium Triplet lie at 8498, 8542 and 8662 \AA\AA.
    
   	%
   	Analysing the behaviour of CaT on the \citet{Vazdekis10} single stellar population models, one can see that its equivalent 
   	width increases with the metallicity of the stars and decreases with the IMF's steepness.
   	%
   	 
	%
	As discussed in Section \ref{sec:IMF}, the IMF slope in the low mass regime is 
	dependent on the initial system mass, in the sense that galaxies with a higher 
	central stellar velocity dispersion (which is a tracer of the galaxy mass) 
	tend to have steeper IMFs. 
	%
	At the same time, we know that the CaT strength anticorrelates with the galaxy mass (Section \ref{sec:metAn}).	
	%
	We can thus assume with a certain degree of confidence that the solution of the ``CaT puzzle'' relies on 
	the link between the CaT equivalent widths and the IMF of the galaxy. 
	%
	In fact, these features are strongly dependent on the surface gravity value of the high mass 
	stars (which dominate the near infrared spectral range) and on their metallicity, 
	so an analysis of their equivalent widths can lead 
	to both the metallicity and the relative number of giant stars in the studied stellar population 
	(i.e. the IMF slope). 
  %
		
	To prove this and, in a second stage, obtain corrected metallicity measurements, we should 
	calibrate a different relation between metallicity and the CaT for different IMF slopes and then 
	compare our results with the metallicities obtained with other methods (i.e. Lick indices). 
	%
	Unfortunately the state-of-art stellar population models do not yet reliably span the entire 
	IMF slope range we need. 
	
	\begin{figure}
%\begin{wrapfigure}{l}{0.5\textwidth}
	\begin{center}
		\includegraphics[width=0.55\textwidth]{./Plots/figK-report.pdf}
	\end{center}
		\caption{Metallicity empirical correction. 
  					%
  					The points represent the metallicity difference at $1~R_{\rm{eff}}$ between our and SAURON 
  					metallicities against the central velocity dispersion $\sigma_{0}$ obtained from the 
  					 HyperLeda%\footnote{http://leda.univ-lyon1.fr} 
  					online archive \citep{Paturel03}. 
  					%
  					The straight line is the fit to the black data points. 
  					%
  					%The red triangles represent NGC 4486 and NGC 4365, which are excluded from the fit being outliers.
  					}\label{fig:IMFcorrection}
	%\end{wrapfigure}
	\end{figure}

	We overcame this problem by adopting an empirical approach. 
	%
	Firstly, we adopted a simple \citet{Salpeter55} IMF to convert the CaT indices into 
	metallicity values without considering the IMF slope dependency. 
	%
	Secondly, we then retrieved the SAURON metallicity maps, obtained using the Lick 
	indices and thus not affected by the CaT dependency on the IMF slope. 
	%
	From these we extracted radial metallicity profiles integrating the values along 
	the galaxy isopothes and, for each galaxy in common between our and SAURON samples, 
	we measured the offset in metallicity at $R = 1~R_{\rm{eff}}$. 
	%
	If the the metallicity of the stellar population would be independent by the number of giant stars 
	over the number of low mass stars, metallicities 
	measured by both the Lick indices and the CaT should be consistent. 
	%
	On the contrary, what we observe is that the metallicity profiles in the two cases show an offset 
	which correlates with the galaxy central velocity dispersion (Figure \ref{fig:IMFcorrection}). 
	%
	This is not surprising if we assume that the CaT depends on both metallicity and IMF (which is 
	linked with the galaxy mass), while Lick indices track only the metallicity. 
  %
		
	Fitting with a simple linear regression the offset values against the galaxy central velocity dispersions we 
	found the relation:
	%
	\begin{eqnarray}
	 \Delta\left[ Z/H\right] =  \left(2.83 \pm 0.35\right) \times \log{\sigma_{0}} - \left(6.05 \pm 1.87 \right)
	\end{eqnarray}
	where $\sigma_{0}$ is the central stellar velocity dispersion (which, again, is a probe for the gravitational potential/total mass) 
	and $\Delta\left[ Z/H\right]$ is the offset between the two profiles measured at $R = 1~R_{\rm{eff}}$. 
	%
%	This because SAURON measurements are not dependent by the IMF of the stars while CaT indices have a deep link with 
%	the mass composition of the stellar population. 
	%
	With this relation, knowing the $\sigma_0$ value of each galaxy in our sample, we extrapolated and applied the metallicity 
	empirical correction to all our measurements. 
	
	\subsection{Kriging}
% Kriging Method
	At this stage, the metallicity data points are randomly scattered within the galaxies because the DEIMOS slits, 
	from which the values are obtained, were designed primarily targeting globular 
	clusters in the galaxies. 
	%
	To obtain metallicity maps we have therefore to interpolate between these metallicity values.
	%
    The method adopted in my project is based on the implementation of a very powerful spatial interpolation technique 
    (Kriging) together with a much larger observational field-of-view. 
	%
	Kriging is a spatial interpolation technique developed in geology \citep{Krige51} and originally used to identify 
	underground mineral deposits from random placed sampling drillings on the surface.	
	%
	The usefullness of this fitting technique, which justifies its use in astronomy, is that it assumes a physical relation 
	between the values of a variable (e.g. metallicity) and the different positions in the explored field. 
	%
%	Kriging is a method for optimal interpolation based on the linear regression against the observed points, 
%	weighted according to the spatial covariance values \citep{Matheron63} and defined as the best linear unbiased 
%	estimator \citep{Cressie88}. 
%	%
%	The initial data points are used to build a ``semivariogram'' which describes the spatial dependence of the probed 
%	variable in the field. 
%	%
%	From this, a relation between all the points in the fields is extrapolated and a reliable estimation of the probed parameter 
%	value can be extracted from the not sampled portions of the map. 
	%
%	In Figure \ref{fig:compareKriging} there is a comparison between the metallicity maps for NGC~5846 obtained 
%	from the same data set adopting three different mapping techniques. 
	%
%	The clear presence of the companion galaxy NGC~5846A at $\approx20~\rm{arcsec}$ from the centre of NGC~5846 
%	can be clearly spotted in the first panel (Kriging) as a blue extended shape. 
	%
%	On the contrary, the natural neighbourhood (Voronoi) map in the second panel and the linear regression 
%	case in the third panel are not able to clearly show this metallicity substructure. 
	%
%	Kriging does not simply interpolate the sampling points in order to create the surface maps, like instead other 
%	mapping methods. 
	%
%	Differently, it extrapolates from the ``semivariogram'' the relation between the map points, assuming that there is 
%	a physical link between their values. 
	
%	
%	\begin{figure}
%    	\begin{center}
%			\includegraphics[width=\textwidth]{./Plots/comparison.pdf}
%		\end{center}
%	    \caption[]{Comparison between different 2D interpolation techniques. 
%	    %
%	    From the same metallicity data set obtained from NGC~5846, a Kriging map (left panel), a 
%	    Voronoi map (central panel) and a linear regression map (right panel) have been obtained. 
%	    % 
%	    The centre of NGC~5846 is marked with a cross. 
%	    %
%	   	The presence of a lower metallicity structure in correspondence of the companion galaxy NGC~5846A 
%	   	about $20~\rm{arcsec}$ south of the main galaxy centre is clearly noticable in the Kriging map, while 
%	   	the Voronoi tiling overestimate its extention and in the linear regression case it not distinguishable from other 
%	   	low metallicity areas around the main galaxy.  
%	   	%
%	   	This figure is best viewed in colour. 
%	    }
%    \label{fig:compareKriging}
%  \end{figure}
	
	\subsubsection{Testing Kriging}
	% Kriging tests
	%
	In order to use a technique mostly adopted in the geological field for astronomical purposes we have to be sure of 
	its reliability. 
	%
	In fact the theory behind this interpolation method implies several assumptions on the data profile. 
	%
	Kriging, in particular, requires that the hypothesis of first and second moments stationarity is satisfied. 
	%
	The first moment stationarity is the condition that each random set of samplings returns the same mean value if the 
	samplings are homogeneously distributed within the field. 
	%
	In general, this is what we expect to find in an astronomical case, where the time variability for the observed parameters is 
	usually on cosmological time scales.
	%
	The second moment stationarity assumption implies that the correlation between the values of two sampled 
	points depends entirely on their relative distance, and not on their locations in the field. 
	%
	Unfortunately, in most of the astronomical applications of the Kriging technique, the spatial distribution of 
	the measured value is not homogeneous (e.g. metallicity has usually a peaked 2D distribution) and it depends on 
	other variables than only distance (e.g. stellar populations, galaxy inclination, etc.). 
	%
	We thus chose to test the quality of a 2D field Kriging reconstruction without this assumption from simulated 
	observations, in order to quantify the reliability of the maps we will obtain from real metallicity data.
	%
	
	Our tests consist of running the Kriging interpolation code using random samplings of mock metallicity 2D 
	distributions. 
	%
	I also associated at each pixel of these maps an estimate metallicity error, accordingly to what I observed in a real 
	case. 
	%
	Testing different numbers of samplings and different metallicity distributions (e.g. changing the position and the PA of the 
	galaxy, adding an extra metallicity shell or an extra galaxy in the field) I compared the retrieved maps with the 
	originals. 
	%
	To address the ability of Kriging to retrieve the original distribution as a whole, for 100 cases I computed the absolute 
	difference in each pixel of the map between the original and the interpolated case in the case of a poor sampling (20 
	data points), an average sampling (50 data points) and a dense sampling (100 data points). 
	%
	I found that in the high and average sampling cases, 68\% of the Kriging pixels have an absolute difference, with respect to 
	the original map, of $\leq 52.7$\% and $\leq 55.7$\% of the associated metallicity error. 
	%
	The low number case, instead, the 68\% of the Kriging pixels have an absolute difference with the original map pixels 
	that is $\leq 75$\% of the associated error. 
	%
	I also verified that the Kriging maps we obtained are good enough to extrapolate trends in the field. 
	%
	With this purpose I extracted the values within a virtual long slit centred on the mock galaxy (or on both the galaxies in 
	the double galaxy case) in both the original and the Kriging retrieved maps. 
	%
	I found that, even in the poorly sampled case, the overall trend is well traced and the mismatching areas are those 
	near the peak of the 2D distribution. 
	%
	From these tests it is possible to conclude that the Kriging method is widely applicable to retrieve the overall 2D 
	distribution of a variable with a high accuracy even with a low number of samplings. 
	%
	However, where the profiles are steep, one needs more samplings to correctly reconstruct the shape of the 
	field. 
	%
	
	
	\subsubsection{Where Kriging Fails}
% Pro and Cons of the Kriging fitting
	Kriging is a powerful estimation technique for the distribution of a parameter in a 2D field. 
	%
	One of the best characteristics of this method is that the points are weighted accordingly with their uncertainties. 
	%
	The possibility to extract from the maps a reliable virtual slit allows us to compare them with the literature data.
	%
	However I verified that in the innermost regions of the galaxies, without a dense enough sampling of the field, 
	Kriging is not able to return the shape of the distribution if the radial gradient is too steep. 
	%
	This is true also with substructures that are spotted most of the times via a random sampling of the field, but which 
	require a specific dense sampling to be properly mapped. 
	%
	
	\subsection{Preliminary results}	% Results: substructures (NGC1407, NGC4365) - Metallicity gradients
	%
	The results I'm going to present in my first paper are:
	\begin{itemize}
		\item	{
					The relation between the IMF steepness and the CaT index (and thus the stellar metallicity). 
					%
					Assuming that the IMF slope correlates with the galaxy mass and that the CaT equivalent width 
					anticorrelates with the IMF steepness (being linked with the giant star), we found an empirical 
					correction for the CaT-retrieved metallicity values which depends on the galaxy central velocity 
					dispersion. 
					%
					To retrieve this correction we compared our measurements with the SAURON metallicities obtained 
					from the Lick indices in 9 common galaxies. 
					%
					Furthermore, we confirmed the link between CaT and galaxy mass, supporting the hypothesis that the 
					solution to the ``CaT puzzle'' relies on the IMF of the system \citep{Cenarro02}. 
					}
	%
		\item	{
					The radial metallicity profiles for 18 early-type galaxies out to $3~R_{\rm{eff}}$. 
					%
					The metallicity values are obtained from the CaT index measured on DEIMOS spectra, reduced to retrieve 
					the stellar component from GC-targeting slits. 
					%
					These profiles are then compared with the inner values and profiles obtained from the literature (where 
					available) and a good match has been found in most of the cases. 
					}
	%
		\item	{
					The astronomical use of the Kriging mapping technique to retrieve the 2D distribution of a 	
					physical variable randomly sampled in the field. 
					%
					We have also tested the ability of retrieving substructures and radial profiles from different 
					samplings and for different 2D distributions, finding the limits of the Kriging method. 
					%
					In particular we noticed that Kriging is not effective in retrieving the values whereas the real gradient is 
					steep, without a very dense samplings (e.g. the metallicity near the galaxy centre). 
					}
	%
		\item	{
%					With the use of the Kriging technique, we obtained the two-dimensional metallicity maps for the galaxies 
%					in our sample %(Figure \ref{fig:NGC5846map}) 
%					together with the estimated uncertainty maps. 
%					%
					From these maps we explored the 2D metallicity profiles, noticing that in most the cases that the metallicity 
					distribution follows in shape the galaxy isophotes. 
					%
%					We also observed that for most the galaxies the metallicity tends to decrease for increasing radii, as expected in a 
%					\texttt{monolithic accretion} scenario. 
					%
%					However, in the other cases the profiles seem flat out to large radii, with the presence of possible substructures 
%					noticeable.
					%
					In several cases we also spotted in this way the possible presence of galaxy substructures with different 
					metallicities, suggesting that the formation of such galaxies could involve mergers. 
					%
					}					
					
					
%
%\begin{figure}
%%\begin{minipage}[t]{.35\textwidth}
%\begin{center}
%\includegraphics[width=0.4\textwidth]{./Plots/NGC5846report.pdf}
%\end{center}
%
%\caption{NGC~5846 Kriging metallicity map. 
%				%
%				The squares show the data points from which the metallicity 2D map 
%				has been obtained. 
%				%
%				The size of the points is inversely proportional with their associated uncertainties. 
%				%
%				The dashed and the dotted lines present the virtual slits along, respectively, the major and the minor axis of 
%				the galaxy. 
%				%
%%				The map and the points are colour-coded according their metallicity values. 
%				%
%				The metallicity ($[Z/H]$) range is shown as a colour bar on the right side of the panel. 
%				%
%				The isocurves of metallicity are obtained from the Kriging map and are presented as solid black contours. 
%				%
%				(This figure is best viewed in colour.)
%	   			%
%				}\label{fig:NGC5846map}
%%\end{minipage}
%\end{figure}	
%	  
  
   
	%					
		\item	{
					From the Kriging maps we extracted the radial metallicity profiles. 
					%
					These profiles are different to long slit profiles because they are measured by averaging the map 
					values along the elliptical annuli defined on the isophotes 
					of the galaxies. 
					%
					These azimuthally averaged radial profiles were compared with the SAURON IFU metallicity profiles.
					%
					Excluding the central region ($R<0.4~R_{\rm{eff}}$), where Kriging is not able to reliably return the 2D profile, 
					we confronted the overlapping radial region up to $1~R_{\rm{eff}}$ in the common galaxies between ours 
					and SAURON samples, finding a good correspondence (rms = 0.29 dex) for their gradients (Figure \ref{fig:gradients}). 
					%
					We also compared these inner metallicity gradients with those measured at $R>1~R_{\rm{eff}}$, finding that 
					the galaxies in our sample seem to present two different behaviours, with several of them presenting a steep outer 
					profile while others have a shallow one. 
					%
					This bimodality could be linked with the formation history of the galaxies, being a shallow metallicity profile 
					a predicted consequence of mergers. 
					}
	\end{itemize}	 
	%
	The last result is still under investigation.
	%
	In the next future we are planning to check if this bimodality in the outer metallicity steepness is linked with the galaxy mass, 
	the environment or other galaxy parameters. 
	%
	
	\begin{figure}
    	\begin{center}
			\includegraphics[width=0.7\textwidth]{./Plots/figU-report.pdf}
		\end{center}
	    \caption[]{Comparison between inner and outer metallicity gradients. 
	    				%
	    				In the left panel the gradients within $1~R_{\rm{eff}}$ are compared between the Kriging profiles (y axis) and 
	    				the SAURON profiles (x axis). 
	    				%
	    				%NGC~5846 is shown as a hollow circle, being an outlier. 
	    				%
	    				The dashed line shows the perfect agreement between the two sets. 
	    				%
	    				The standard deviation with respect to the perfect match is $\sigma = 0.29$ dex. 
	    				%
	    				In the right panel the inner metallicity Kriging gradients (y axis) are presented against the 
	    				Kriging gradients up to $3~R_{\rm{eff}}$ (x axis). 
	    				%
	    				%The standard deviation with respect to the match between inner and outer gradients is $\sigma =0.52$.
	    }
    \label{fig:gradients}
  \end{figure}
	
	
\section{Second thesis project: extending the metallicity profiles to include red GCs}
	The bimodality in the colour histogram of globular clusters in early-type galaxies is very well known and demonstrated for a large 
	sample of objects \citep{Kundu01, Larsen01}. 
	%
	\citet{Usher12} show that this bimodality is due to the metallicity difference between the two globular cluster 
	subpopulations. 
	%
	It has also been shown by \citet{Pota13} that the red subpopulation in most of their galaxy sample are 
	kinematically coupled with the galaxy's. 
	%
	Therefore, in this project we will study together the 2D metallicity fields of the red globular
	cluster systems in addition to the galaxy stars, thereby extending the techniques developed in my first project to larger radii. 
	%
	We will probe distances up to $10~R_{\rm{eff}}$ adopting the the globular cluster metallicity measurements presented 
	in \citet{Usher12} updated with the most recent DEIMOS observations carried out in the last 12 months.
	%
	The first results in this sense are promising. 
	%
	As it is possible to see in Figure \ref{fig:SkimsGC}, the radial trend of red globular clusters for NGC~5846 appears 
	consistent with the innermost profiles from the literature and the outer stellar measures obtained using the reduction 
	technique adopted in the first project. 
	%	
  
\begin{figure}
\centering
%\begin{minipage}[t]{.4\textwidth}
%	\begin{figure}
    	\begin{center}
			\includegraphics[width=0.4\textwidth]{./Plots/met.pdf}
		\end{center}
	    \caption[]{NGC~5846 radial metallicity profile. 
	    %
	    The red star symbol shows the inner metallicity value measured by \citet{Tang09}, while the green 
	    line is the profile extracted from the SAURON metallicity map. 
	    %
	    The black points show the metallicities obtained from the DEIMOS stellar spectra and the red squares are the metallicities of 
	    the red globular cluster subpopulation measured by \citet{Usher12}. 
	    %
	    The typical error for the globular cluster metallicities is shown as a red solid line, while the red dashed line 
	    shows the weighted fit for the globular cluster values. 
	    %
		(This figure is best viewed in colour.) 
	   	%
%	    This is consistent with the inner trends of both SAURON and my datasets. 
	    }
    \label{fig:SkimsGC}
  \end{figure}
%\caption{Caption}\label{label-a}
%\end{minipage}\qquad

  
  
  
  
  
  
  
\section{Third thesis project: detailed analysis of the metallicity profiles of NGC~1407 and NGC~4365}  
%
  The third project will focus on the analysis of the metallicity distributions for the two 
  particular galaxies NGC~1407 and NGC~4365.
  %
  The first is a massive system which dominates the Eridanus A group \citep{Brough06}. 
  %
  NGC~1407 is a galaxy with a steep metallicity gradient in the inner $R_{\rm{eff}}$ and a kinematically 
  decoupled core at the centre \citep{Spolaor08a}. 
  %
  The presence of both these features is not compatible with a pure \texttt{monolithic accretion} scenario or a 
  formation driven by dry mergers. 
  %
%  While the first feature is not compatible with a dissipativeless
%  The presence of the strong metallicity gradient suggests the absence of many mergers in the galaxy formation history, this is 
%  in apparent contradiction with the presence of the decoupled core. 
  %
  Furthermore, from the extended stellar radial profiles and 2D maps obtained in the first project of my thesis, the presence 
  of a possible high metallicity ring-shaped structure at $R\approx0.6~R_{\rm{eff}}$ is noticeable (Figure \ref{fig:N1407}).  
  %
  This could be a clear sign of a past accretion of star-forming gas from a companion galaxy, 
  but more analysis is needed in order to understand the evolution of this galaxy. 
  %
  
  %
  For similar reasons, another case worthy of further studies is the giant elliptical NGC~4365 in the Virgo Cluster. 
  %
  This galaxy has already been proven to be a very interesting object, with probable multiple star formation episodes (\citealt{Blom12} 
  and references therein), a kinematically distinct core and a weird global kinematic \citep{Davies01}. 
  %
  A stream of stars from the nearby galaxy NGC~4342 \citep{Blom13} suggests that NGC~4365 is still undergoing 
  merger processes and thus can be considered as a good case to test metallicity mapping. 


	\begin{figure}
    	\begin{center}
			\includegraphics[width=0.5\columnwidth]{./Plots/NGC1407-report.pdf}
		\end{center}
	    \caption[]{NGC~1407 Kriging metallicity map. 
				%
				The squared points show the data points from which the metallicity 2D map 
				has been obtained. 
				%
				The size of the points is inversely proportional with the associated uncertainties. 
				%
				The map and the points are colour-coded according their metallicity values. 
				%
				The metallicity ($[Z/H]$) range is shown as a colour bar on the right side of the panel. 
				%
				The isocurves of metallicity are obtained from the Kriging map and are presented as solid black lines. 
				%
				The presence of a high metallicity substructure surrounding the galaxy centre is noticeable at a distance 
				of $\approx 40~\rm{arcsec}$ from the galaxy centre. 
				%
			   	This figure is best viewed in colour. 
			   	}
    \label{fig:N1407}
  \end{figure}
	



\section{Meetings, schools and conferences attended}
	In the last 12 months I have attended the following scientific meetings, conferences and school:
	%
	\begin{itemize}
%		\item{Swinburne Keck Science workshop}, 26$^{\rm{th}}$ - 28$^{\rm{th}}$ March 2012, Swinburne University.
		%
		\item{Harley Wood Winter School}, 28$^{\rm{th}}$ June - 1$^{\rm{st}}$ July 2012, Blue Mountains (NSW).
		%
		\item{ASA Annual Scientific Meeting}, 1$^{\rm{st}}$ - 6$^{\rm{th}}$ July 2012, UNSW, Sydney (NSW).
		%
		\item{Publishing With Impact workshop}, 10$^{\rm{th}}$ - 11$^{\rm{th}}$ December 2012, Swinburne University.
		%
		\item{Astroinformatics school}, 18$^{\rm{th}}$ - 20$^{\rm{th}}$ February 2012, BQ, Brisbane (QL).  
	\end{itemize}
	%
	
\section{Extra Publications}
	%
	\begin{itemize}
	%
	\item{\textbf{Pastorello}, \textbf{N.} et al., 2013}, ``The planetary nebulae population in the nuclear regions of M31: the SAURON view'', MNRAS, 420, 1219.
	%
	\item{Corsini, E. M., Mend\'ez-Abreu, J., \textbf{Pastorello}, \textbf{N.} et al., 2012}, ``Polar bulges and polar nuclear discs: the case of NGC~4698'', MNRAS, 423, 79.	
	%
\end{itemize}		



\section{Future plans}
\begin{itemize}
\item{June 2013:} presenting my work at the ``Small Stellar Systems in Tuscany'' conference in Tuscany, Italy and at the 
                     "The Physical Link between Galaxies and their Halos'' conference in Garching, Germany. 
\item{July 2013:} presenting my work at the ``ASA Annual Scientific meeting'' in Melbourne, Australia.
\item{August 2013:} submission of first paper.
\item{September 2013:} start second paper.
\item{February 2013:} second paper submission.
\item{March 2014:} start third paper.
\item{July 2014:} 30 month review. 
\item{September 2014:} third paper submission.
\item{October 2014:} start thesis writing.
\item{December 2014:} thesis submission.

\end{itemize}

\section{Thesis plan}

My preliminary thesis plan is the following:
\begin{itemize}
\item{First Chapter: } Introduction, Galaxy Formation and Evolution models, The Initial Mass Function issue.
\item{Second Chapter:} Mapping the Metallicity of stars in Early-Type galaxies.
\item{Third Chapter:} The Connection between red Globular Clusters and Galaxy stars Metallicity Profiles.
\item{Fourth Chapter:} Analysis of two Peculiar Cases: NGC~1407 and NGC~4365.
\item{Fifth Chapter:} Conclusions, Big picture and connections with my work.
\item{Appendix:} Application of the Kriging Method in Astronomy.
\end{itemize}

